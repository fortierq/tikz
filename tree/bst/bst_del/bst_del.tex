\documentclass[convert={outfile=\jobname.png}]{standalone}
\usepackage{draw}

\begin{document}

\begin{tikzpicture}[every node/.style = {minimum size=2em, vertex}]] 
    \node[bottom color=white!30!red, minimum size=2em] {?}
    child { node {1} 
        child { node {-2} }
        child { node {3} 
            child { node {2} }
            child { node {4} } } }
    child { node {11} 
        child { node {8} 
            child { node  {7} } child[missing] {node {}}}
        child { node {23} } };
\end{tikzpicture}

    \begin{tikzpicture}[tree layout, level distance=12mm, sibling distance=3.8em,
    every node/.style = {minimum size=2em, shape=circle, draw, align=center, top color=white, bottom color=blue!20}]]
    \node[bottom color=white!30!red, minimum size=2em] {?}
    child { node {1} 
        child { node {-2} }
        child { node {3} 
            child { node {2} }
            child { node {4} } } }
    child { node {11} 
        child { node {8} 
            child { node {7} } child[missing] {node {}}}
        child { node {23} } };
    \end{tikzpicture}
\end{center}
\end{frame}

\begin{frame}[fragile]
\frametitle{Arbre binaire de recherche (ABR)}
Supprimer un élément \verb|e| (ici 6) dans un ABR:
\begin{center}
    \begin{tikzpicture}[tree layout, level distance=12mm, sibling distance=3.8em,
    every node/.style = {minimum size=2em, shape=circle, draw, align=center, top color=white, bottom color=blue!20}]]
    \node[bottom color=white!30!red] (r) {?}
    child { node {1} 
        child { node {-2} }
        child { node {3} 
            child { node {2} }
            child { node[shape=circle, draw, align=center, top color=white, bottom color=black!20!green] (f) {4} } } }
    child { node {11} 
        child { node {8} 
            child { node {7} } child[missing] {node {}}}
        child { node {23} } };
    \draw[->, >=latex, thick] (-.8,-3.5) to[bend right = 20] (0,-.5);
    \end{tikzpicture}
\end{center}
\end{frame}


\begin{frame}[fragile]
\frametitle{Arbre binaire de recherche}
Supprimer un élément \verb|e| (ici 6) dans un ABR:
\begin{center}
    \begin{tikzpicture}[tree layout, level distance=12mm, sibling distance=3.8em,
    every node/.style = {minimum size=2em, shape=circle, draw, align=center, top color=white, bottom color=blue!20}]]
    \node[shape=circle, draw, align=center, top color=white, bottom color=black!20!green] {4}
    child { node {1} 
        child { node {-2} }
        child { node {3} 
            child { node {2} } child[missing] {node {}}} }
    child { node {11} 
        child { node {8} 
            child { node {7} } child[missing] {node {}}}
        child { node {23} } };
    \end{tikzpicture}
\end{center}
\end{frame}

\begin{tikzpicture}[level distance=12mm, sibling distance=3.8em,
    every node/.style = {vertex}]]
    \node[bottom color=white!30!red] {?}
    child { node {1} 
        child { node {-2} }
        child { node {3} 
            child { node {2} }
            child { node {4} } } }
    child { node {11} 
        child { node {8} 
            child { node {7} } child[missing] {node {}}}
        child { node {23} } };
\end{tikzpicture}

% \begin{tikzpicture}[every node/.style={vertex}, level 1/.style={sibling distance=3.5cm}, level 2/.style={sibling distance=1.8cm}, level 3/.style={sibling distance=1cm}]
%     \node (r) {$4$}
%     child {
%         node (g) {$3$}
%         child {
%             node (gg) {$1$}
%             child[] {node {$0$}}
%             child {node {$2$}}
%         }
%         child[missing] {
%             node (gd) {$3$}
%         }
%     }
%     child {
%         node (d) {$6$}
%         child {node (dg) {$5$}}
%         child {
%             node (dd) {$8$}
%             child[missing] {node {$7$}}
%             child {node {$9$}}
%         }
%     };
% \end{tikzpicture}

\end{document}
